\documentclass[
    a4paper,
    12pt,
    parskip=half,
    headings=standardclasses,
    footskip=0pt,
    footlines=1,
    headheight=80in
]{scrartcl}

\sloppy
\usepackage[utf8]{inputenc}
\usepackage[T1]{fontenc}
\usepackage{microtype}

\usepackage{amsmath,amssymb,amsthm}
\usepackage{minted}

\usepackage{hyperref}
\usepackage{tcolorbox}

\usepackage{newpxtext,newpxmath}

\newcommand{\lgrey}{black!5}

\begin{document}

\begin{center}
    {\LARGE\textbf{Automated Bidding Challenge}}\\[6pt]
    Time limit: 10 seconds, Memory limit: 50 MB
\end{center}

\paragraph*{Problem}

For this challenge, you will be playing a round-based auction game against all
other participants simultaneously.

You start off with $10^5$ coins. In each round, you bid a number of coins. Everyone loses the number of coins they bid.
The player with the highest bid wins the round and gets 1 point. In case of a tie, all participants with the highest bid
split the point evenly.

After $10^3$ rounds, all players are ranked by the
number of points they earned (for more details, see \hyperref[scoring]{Scoring}).

There will be 3 passes of the game as some solutions may use
randomized strategies.

\paragraph*{Interaction} All interaction is done via standard input and output.

The first line of input contains a single integer $n$ ($1 \leq n \leq 10^3$),
the number of opponents.

Then for each round, you submit a bid by printing a single integer $b$ ($0 \leq
    b \leq 10^5$) to standard output\footnote{Remember to flush the output!}. After
making a bid, you will receive a line of input containing $n$ space separated
integers $b_1, \ldots, b_n$ ($-1 \leq b_i \leq 10^5$), the bid of your $i$-th
opponent. A bid of $-1$ means that the opponent made an invalid bid or hit the
time or memory limit. You are not required to read in the bids of your opponents after
making your last bid.

If you make an invalid bid or hit the time or memory limit, you will be
disqualified for the rest of the game. Your earned points will still remain valid.

See \hyperref[sample-programs]{Sample Programs} for an exemplary implementation.

\paragraph*{Example interaction} Below is an example interaction between four participants over three rounds
each starting with $6$ coins. On the left is the input given to the program, on the right is the output of the program.

\begin{tcolorbox}[colback=\lgrey,fontupper=\small\ttfamily]
3

\hfill0

3 2 4

\hfill4

2 3 -1

\hfill2

1 1 -1
\end{tcolorbox}

\clearpage

\subsection*{Appendix}

\paragraph*{Sample programs}\label{sample-programs}

Here are two sample programs that play the game using a simple ``copy the
winner'' strategy.

\begin{minted}[bgcolor=\lgrey]{python}
# copycat.py
n = int(input())
coins = 10**5
rounds = 10**3

other_bids = [0]*n

for round in range(rounds):
    my_bid = max(0, min(coins, max(other_bids)))
    coins -= my_bid
    print(my_bid, flush=True)
    
    other_bids = [int(x) for x in input().split()]
\end{minted}

\begin{minted}[bgcolor=\lgrey]{cpp}
// copycat.cpp
#include <bits/stdc++.h>
using namespace std;

int main() {
    cin.tie(0)->sync_with_stdio(0);
    int n; cin >> n;
    int coins = 1e5, rounds = 1e3;
    vector<int> other_bids(n);
    for (int round = 0; round < rounds; round++) {
        int my_bid = clamp(
            *max_element(other_bids.begin(), other_bids.end()),
            0,
            coins
        );
        coins -= my_bid;
        cout << my_bid << endl;
        for (int &bid : other_bids) cin >> bid;
    }
}
\end{minted}

\paragraph*{Scoring}\label{scoring}

Submissions are ranked by the following criteria, in order:
\begin{enumerate}
    \item The number of points.
    \item The number of rounds being not disqualified.
    \item The number of coins left.
\end{enumerate}

\paragraph*{System Details} All submissions are run on a single machine with the
following specifications:
\begin{itemize}
    \item Intel i7-11370H
    \item 16 GB RAM
    \item Debian GNU/Linux 11 (bullseye)
\end{itemize}

You may submit your solution as a single source file in any of the following
languages:
\begin{enumerate}
    \raggedright
    \item C++ 20 (g++ 10.2.1)\\[4pt]compiled as \texttt{g++ -std=gnu++20 -x c++ -Wall -O2 -static -pipe -o \$1 "\$1.cpp" -lm}
    \item Python 3.7.10 (PyPy 7.3.5)\\[4pt]run as \texttt{pypy3 \$1}
\end{enumerate}
You may only use the standard library of your chosen language.
You may not use
\begin{enumerate}
    \item any other libraries or packages.
    \item any other files or resources.
    \item any form of inter-process communication.
    \item the internet.
\end{enumerate}

\end{document}
